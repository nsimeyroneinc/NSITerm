\documentclass[a4paper, 11pt]{article}

\title{
  \rule{\linewidth}{0.8pt}
  \Large{\scshape{\textbf{Numérique et Sciences Informatiques}}}

  \vspace{0.5cm}
  \large{\textit{Épreuve type BAC}}

  \rule[10pt]{\linewidth}{0.8pt}
    }
\author{}
\date{20 février 2023}

%-------------------------------------------------------
% Les packages
%-------------------------------------------------------
\usepackage[T1]{fontenc}
\usepackage[utf8]{inputenc}
% \usepackage{multicol} % les colonnes
\usepackage[french]{babel}
\addto\captionsfrench{\def\tablename{Tableau}}
\addto\captionsfrench{\def\figurename{Figure}}
\usepackage{lmodern}
\usepackage{hyperref} % liens hypertextes
% \usepackage{graphicx} % insertion d'images
\usepackage{booktabs} % Les tableaux
\usepackage{amssymb} % symboles maths
\usepackage{amsmath} % symboles maths
% \usepackage{blindtext} % faux texte
\usepackage{float} % positionnement des tableaux et autres flottants
\usepackage{multirow} % lignes fusionnées dans les tableaux 
\usepackage{lastpage} % compter les pages

%-------------------------------------------------------
% Figure geogebra -> tikz
%-------------------------------------------------------
\usepackage{pgfplots}
\pgfplotsset{compat=1.15}
\usepackage{mathrsfs}
\usetikzlibrary{arrows, positioning, snakes, calc, shapes.misc, decorations.shapes, decorations.markings, backgrounds}

%-------------------------------------------------------
% Dimension de la page
%-------------------------------------------------------
%\usepackage[a4paper, total={17cm, 25cm}, includehead, headsep=12pt]{geometry} %taille de la page et marges
\usepackage[a4paper,vmargin=1.5cm,hmargin=1.5cm,tmargin=2cm,bmargin=1.5cm]{geometry}
%-------------------------------------------------------
% Longueurs personnelles
%-------------------------------------------------------
% \setlength{\textfloatsep}{0.1cm}
\setlength{\parskip}{6pt}
\setlength{\parindent}{15pt}
% \setlength{\textfloatsep}{10pt plus 1.0pt minus 2.0pt}
% \setlength{\floatsep}{10pt plus 1.0pt minus 2.0pt}
% \setlength{\intextsep}{10pt plus 1.0pt minus 2.0pt}
\newlength{\aftercode}
\setlength{\aftercode}{20pt}

%-------------------------------------------------------
% Les entêtes et pied de pages
%-------------------------------------------------------
\usepackage{fancyhdr}
\pagestyle{fancy}
\fancyhf{}
\fancyhead[L]{\scriptsize Terminale - NSI}
\fancyhead[R]{\scriptsize Bac Blanc}
\fancyfoot[C]{\scriptsize - $\thepage$/\pageref{LastPage} -}
\renewcommand{\headrulewidth}{0.3pt}
\renewcommand{\footrulewidth}{0.3pt}

%-------------------------------------------------------
% La numérotation des questions et des listes
%-------------------------------------------------------
\usepackage[shortlabels]{enumitem}
\setlist[itemize,1]{label=\textbullet} % le label des listes de niveau 1 est un bullet
\setlist[itemize,2]{label=$\circ$} % le label des listes de niveau 2 est un bullet vide
\setlist[itemize,3]{label=$\diamond$} % le label des listes de niveau 3 est un diamant


%-------------------------------------------------------
% L'environnement Exercice
%-------------------------------------------------------
\newcounter{exercice}
\newenvironment{Exercice}[1][]{
    \refstepcounter{exercice}
    \par
    \noindent
    {\large
    \underline{\textit{Exercice~\theexercice~:}}~#1}
    % \vspace{\parskip}
    \par}
    {\vspace{1em}}

%-------------------------------------------------------
% Dossiers contenant les images
%-------------------------------------------------------
\graphicspath{ 
	{images/}
}

%-------------------------------------------------------
% L'environnement Question
%-------------------------------------------------------
\newcounter{question}[exercice]
\newenvironment{Question}{
    \refstepcounter{question}
    \begin{list}{}{%
      \setlength{\labelindent}{\parindent}
      \setlength{\leftmargin}{2\parindent}
      \setlength{\parsep}{\parskip} % Espacement vertical des sous-paragraphes
    }
    \item[\medskip\textbf{\thequestion.}]}
    {\end{list}}

%-------------------------------------------------------
% L'environnement subQuestion
%-------------------------------------------------------
\newcounter{subquestion}[question]
\newenvironment{subQuestion}{
    \refstepcounter{subquestion}
    \begin{list}{}{%
      \setlength{\leftmargin}{12pt}
      \setlength{\labelwidth}{8pt}
      \setlength{\labelsep}{4pt}
      \setlength{\listparindent}{0em}
      \setlength{\itemindent}{0em}
      \setlength{\parsep}{\parskip} % Espacement vertical des sous-paragraphes
    }
    \item[\medskip\textbf{\alph{subquestion}.}]}
    {\end{list}}
  

%-------------------------------------------------------
% Création de couleurs
%-------------------------------------------------------
\usepackage{xcolor}
\definecolor{bg_gris}{rgb}{0.95,0.95,0.95}

%-------------------------------------------------------
% Mise en forme du code
%-------------------------------------------------------
\usepackage[newfloat=false]{minted}
\setminted{autogobble, tabsize=4, breaklines, breakafter=_, linenos, numberblanklines=false, resetmargins, frame=leftline}
% Environnement python code
\newminted{python}{} % permet d'utiliser \begin{pythoncode} pour écrire du python

% Environnement mintinlinepy
\newmintinline[mintinlinepy]{python}{breaklines, breakafter=_} % permet d'utiliser \mintinlinepy{...} pour écrire du python en ligne
    
% Environnement textcode
\newminted{text}{} % permet d'utiliser \begin{textcode} pour écrire du code en français
% \BeforeBeginEnvironment{textcode}{\vspace{-2\parskip}\begin{listing}[H]}
% \AfterEndEnvironment{textcode}{\end{listing}\vspace{-\aftercode}}

% Environnement textcode
\newmintinline[mintinlinetext]{text}{breaklines, breakafter=_} % permet d'utiliser \mintinlinetext{...} pour écrire du code en ligne en français


% Environnement de code sur plusieurs pages
\usepackage{caption}
\newenvironment{longcode}{\captionsetup{type=listing}}{}

%-------------------------------------------------------
% Commandes personnelles
%-------------------------------------------------------
\newcommand{\python}{\texttt{python}}
% \renewcommand{\thesection}{\Alph{section}.}
%-------------------------------------------------------

% Le document
%-------------------------------------------------------
\begin{document}

%-------------------------------------------------------
% La page de titre
%-------------------------------------------------------
\maketitle
\thispagestyle{fancy}

% \vspace*{-2cm}

\begin{center}
  Durée de l'épreuve : 2 heures 00
\end{center}

\vspace*{1cm}

L'usage de la calculatrice et du dictionnaire n'est pas autorisé.

Dès que ce sujet vous est remis, assurez-vous qu'il est complet.
Ce sujet comporte \pageref{LastPage} pages numérotées de 1/\pageref{LastPage} à \pageref{LastPage}/\pageref{LastPage}.

\vfill
\makebox[\textwidth]
{Nom :\enspace\hrulefill}

\vspace{0.5cm}

\makebox[\textwidth]
{Classe:\enspace\hrulefill}

\vfill

\newpage
% %-------------------------------------------------------
% % La table des matières
% %-------------------------------------------------------
% \setcounter{tocdepth}{3}
% \tableofcontents
% \newpage


\begin{Exercice}[\textit{Base de données}\hfill 8 points]

  Dans cet exercice, on pourra utiliser les mots clés suivants du langage \mintinlinetext{SQL} : \mintinline{SQL}{SELECT, FROM, WHERE, JOIN, ON, INSERT INTO, UPDATE,
 SET, VALUES, COUNT, OR, AND, LIKE}.

  Les articles d'un journal sont recensés dans une base de données \mintinline{SQL}{Journal}. Cette base contient des informations sur chaque article (titre,
  auteur et date de parution). Chaque auteur est caractérisé par son nom et son prénom. On associe de plus à certains articles des thèmes.

  Le schéma relationnel de la base de données est donné ci-dessous :
  \begin{itemize}
    \item \mintinline{SQL}{Articles} (\underline{idArticle}, \mintinline{SQL}{titre, #auteur, dateParution)}
    \item \mintinline{SQL}{Auteurs} (\underline{idAuteur} \mintinline{SQL}{, nom, prenom)}
    \item \mintinline{SQL}{Themes} (\underline{idTheme} \mintinline{SQL}{, theme)}
    \item \mintinline{SQL}{Traitements} (\underline{idTraitement} \mintinline{SQL}{, #article, #theme)}
  \end{itemize}

  Dans ce schéma, les clés primaires sont soulignées et les clés étrangères sont précédées du symbole \mintinline{SQL}{#}.

  L'attribut \mintinline{SQL}{auteur} de la relation \mintinline{SQL}{Articles} est une clé étrangère faisant référence
  à l'attribut \mintinline{SQL}{idAuteur} de la relation \mintinline{SQL}{Auteurs}.

  L'attribut \mintinline{SQL}{article} de la relation \mintinline{SQL}{Traitements} est une clé étrangère faisant référence
  à l'attribut \mintinline{SQL}{idArticle} de la relation \mintinline{SQL}{Articles}.

  L'attribut \mintinline{SQL}{theme} de la relation \mintinline{SQL}{Traitements} est une clé étrangère faisant référence
  à l'attribut \mintinline{SQL}{idTheme} de la relation \mintinline{SQL}{Themes}.

  Tous les attributs dont le nom est préfixé par \mintinline{SQL}{id} sont des nombres entiers ainsi que l'attribut \mintinline{SQL}{dateParution} (les dates
  sont notées au format \mintinline{SQL}{aaaaMMjj}, par exemple \mintinline{SQL}{20220103} pour le 3 janvier 2022). Les autres attributs sont des chaînes de caractères.

  Les entrées de la relation \mintinline{SQL}{Traitements} correspondent aux thèmes abordés par des articles. Un article peut aborder 0, 1 ou plusieurs thèmes.

  \begin{Question}
    Expliquer pourquoi il n'est pas possible d'insérer une entrée dans la relation \mintinline{SQL}{Articles} si les relations
    \mintinline{SQL}{Auteurs} et \mintinline{SQL}{Themes} sont vides.
  \end{Question}

  On fournit les extraits suivants de chacune des relations.

  \begin{table}[H]
    \begin{center}
      \texttt{
        \begin{tabular}{llll}
          \multicolumn{2}{l}{Articles}                                 \\
          \toprule
          idArticle & titre                        & auteur & dateParution     \\
          \midrule
          1         & Présidence française de l'UE & 2      & 20220112 \\
          2         & Les bleues en demies !       & 3      & 20220110 \\
          \bottomrule
        \end{tabular}
      }
    \end{center}
  \end{table}

  \begin{minipage}{0.45\textwidth}
    \begin{table}[H]
      \begin{center}
        \texttt{
          \begin{tabular}{lll}
            \multicolumn{3}{l}{Auteurs}   \\
            \toprule
            idAuteur & nom     & prenom  \\
            \midrule
            1         & Londres & Alfred  \\
            2         & Jeraus  & Jean    \\
            3         & Zola    & Étienne \\
            \bottomrule
          \end{tabular}
        }
      \end{center}
    \end{table}
  \end{minipage}
  \begin{minipage}{0.45\textwidth}
    \begin{table}[H]
      \begin{center}
        \texttt{
          \begin{tabular}{ll}
            \multicolumn{2}{l}{Themes} \\
            \toprule
            idTheme & themes           \\
            \midrule
            1       & Politique        \\
            2       & Sport            \\
            3       & Europe           \\
            4       & Handball         \\
            \bottomrule
          \end{tabular}
        }
      \end{center}
    \end{table}
  \end{minipage}

  \begin{table}[H]
    \begin{center}
      \texttt{
        \begin{tabular}{lll}
          \multicolumn{3}{l}{Traitements} \\
          \toprule
          idTraitement & article & theme  \\
          \midrule
          1            & 1       & 1      \\
          2            & 1       & 3      \\
          3            & 2       & 2      \\
          4            & 2       & 3      \\
          \bottomrule
        \end{tabular}
      }
    \end{center}
  \end{table}

  \begin{Question}
    L'article nommé \og \textit{Les bleues en demies !} \fg\ portait sur la coupe d'Europe de handball. Écrire une requête \mintinline{SQL}{SQL} permettant d'ajouter le
    thème \mintinline{SQL}{handball} à cet article dans la table (ou relation) \mintinline{SQL}{Traitements}.
  \end{Question}

  \begin{Question}
    Le journaliste \og Jean Jèraus \fg\ tient à ce que son nom soit écrit avec l'accent grave. Ce n'est pas le cas pour l'instant dans la base de données.
    Écrire la requête \mintinline{SQL}{SQL} permettant de mettre à jour les informations afin de satisfaire sa demande.
  \end{Question}

  \begin{Question}
    Pour chacun des items suivants, écrire une requête \mintinline{SQL}{SQL} permettant d'extraire les informations demandées.
    \begin{subQuestion}
      Le titre des articles parus après le $1^{er}$ janvier 2022 inclus.
    \end{subQuestion}
    \begin{subQuestion}
      Le titre des articles écrits par l'auteur Étienne Zola (on sait que son \mintinline{SQL}{idAuteur} est le \mintinline{SQL}{3}).
    \end{subQuestion}
    \begin{subQuestion}
      Le nombre d'articles écrits par l'auteur Jacques Pulitzer (présent dans la table \mintinline{SQL}{Auteurs} mais on ne connaît pas son \mintinline{SQL}{idAuteur}).
    \end{subQuestion}
    \begin{subQuestion}
      Les dates de parution des articles traitant du thème \og Sport \fg.
    \end{subQuestion}
  \end{Question}

\end{Exercice}

\begin{Exercice}[\textit{Arbres binaires équilibrés}\hfill 12 points]
  On s'intéresse dans cet exercice aux arbres binaires \textbf{équilibrés}.

  On dit qu'un arbre binaire est équilibré si, quel que soit le nœud considéré, les hauteurs des sous-arbres issus de ses fils gauche
  et droit ne diffèrent au maximum que de 1. Dans la figure \ref{equilibre}, l'arbre de gauche est équilibré alors que celui de droite ne l'est pas.

  On considère que la hauteur d'un arbre réduit à sa racine vaut 1. La hauteur d'un arbre vide vaut donc 0.

  \begin{figure}[!ht]
    \begin{center}
      \begin{tikzpicture}[level distance=0.8cm]
        \tikzstyle{every node}=[draw, circle, minimum size=3mm, inner sep=0mm]

        \node at (0,0) {} [level 1/.style={sibling distance=4cm},
          level 2/.style={sibling distance=2cm},
          level 3/.style={sibling distance=1cm}]
        child {node {}
            child {node {}
                child {node[draw=none] {} edge from parent [draw=none]}
                child {node {}}
              }
            child {node {}
                child {node {}}
                child {node {}}
              }
          }
        child {node {}
            child {node {}
                child {node {}}
                child {node[draw=none] {} edge from parent [draw=none]}
              }
            child {node {}
              }
          };


        \node at (9,0) {} [level 1/.style={sibling distance=4cm},
          level 2/.style={sibling distance=2cm},
          level 3/.style={sibling distance=1cm}]
        child {node {}
            child {node {}
                child {node[draw=none] {} edge from parent [draw=none]}
                child {node {}}
              }
            child {node {}
                child {node {}}
                child {node {}}
              }
          }
        child {node[fill=black] {}
            child {node {}
                child {node {}}
                child {node[draw=none] {} edge from parent [draw=none]}
              }
            child {node[draw=none] {} edge from parent [draw=none]}
          };
      \end{tikzpicture}
    \end{center}
    \caption{Un arbre binaire équilibré (à gauche) et un non-équilibré (à droite)}
    \label{equilibre}
  \end{figure}

  \noindent{\textbf{Partie A}}
  \setcounter{question}{0}

  Afin de déterminer si un arbre est équilibré ou non, on calcule la \textbf{balance} de chaque nœud. Cette grandeur est égale à la différence des hauteurs de son fils
  droit et de son fils gauche. Dans la figure \ref{equilibre}, le nœud colorié en noir a une balance de $-2$. En effet, il n'a pas de fils droit et son fils gauche a une
  hauteur de 2 : sa balance vaut donc $0-2=-2$.

  Un arbre binaire est équilibré si la balance de chacun de ses nœuds vaut $-1$, $0$ ou $1$.

  \begin{Question}
    \begin{subQuestion}
      On considère l'arbre de la figure \ref{arbreQ1}. Compléter la figure sur cet énoncé en indiquant à côté de chaque nœud sa balance.
    \end{subQuestion}
    \begin{subQuestion}
      Cet arbre est-il équilibré ?
    \end{subQuestion}
  \end{Question}

  \begin{figure}[H]
    \begin{center}
      \footnotesize{

        \begin{tikzpicture}[level distance=1cm,level 1/.style={sibling distance=4cm},
            level 2/.style={sibling distance=2cm},
            level 3/.style={sibling distance=1.5cm}]
          \tikzstyle{every node}=[draw, circle, minimum size=7mm, inner sep=1mm]

          \node at (0,0) {9} []
          child {node {6}
              child {node {3}
                  child {node {2}}
                  child {node[draw=none] {} edge from parent [draw=none]}
                }
              child {node {7}
                }
            }
          child {node {15}
              child {node[draw=none] {} edge from parent [draw=none]}
              child {node {18}
                  child {node {16}}
                  child {node[draw=none] {} edge from parent [draw=none]}
                }
            };

        \end{tikzpicture}
      }
    \end{center}
    \caption{}
    \label{arbreQ1}
  \end{figure}

  On implémente les arbres binaires à l'aide de listes \python. Dans ce cadre :
  \begin{itemize}
    \item On place la valeur \texttt{0} dans la cellule d'indice $0$. \textbf{Cette valeur sera ignorée dans les différents traitements}
    \item la valeur de la racine de l'arbre est dans la cellule d'indice $1$
    \item si l'on considère un nœud dont la valeur est stockée dans la cellule $n$ :
          \begin{itemize}
            \item la valeur de son fils gauche est stockée dans la cellule $2n$
            \item la valeur de son fils droit est stockée dans la cellule $2n+1$
          \end{itemize}
    \item si un nœud est vide, on stocke une valeur absurde dans le tableau (\mintinlinepy{None} en \python)
    \item le tableau se termine à la dernière valeur non-vide
  \end{itemize}

  \newcounter{index}
  \begin{figure}[!ht]
    \centering
    \footnotesize{
      \begin{tikzpicture}
        \setcounter{index}{0}
        \coordinate (s) at (0,0);
        \foreach \num in {0,9,6,15,3,7,/,18,2,/,/,/,/,/,16}{
            \node[minimum size=8mm, draw, rectangle] at (s) (\theindex) {\num};
            \node at ($(s)+(0,0.6)$) (i\theindex) {\footnotesize \theindex};
            \stepcounter{index}
            \coordinate (s) at ($(s) + (0.8,0)$);
          }
        \draw [->] (i2.north) to [out=45,in=150] (i4.north);
        \draw [->] (i2.north) to [out=45,in=150] (i5.north);
        \draw [->] (i7.north) to [out=30,in=150] (i14.north);

        \node[draw=none, anchor=north, rectangle] at ($(7)+(0.3,-1)$) {
          \begin{minipage}{14.7cm}
            \normalsize{
              \mintinlinepy{[0, 9, 6, 15, 3, 7, None, 18, 2, None, None, None, None, None, 16]}
            }
          \end{minipage}
        };
      \end{tikzpicture}
    }
    \caption{Représentation de l'arbre de la figure \ref{arbreQ1} dans un tableau et une liste \python}
  \end{figure}

  \begin{Question}
    \begin{subQuestion}
      Donner la liste \python\ représentant l'arbre de la figure \ref{arbreQ2}.

      \begin{figure}[!ht]
        \centering
        \footnotesize{

          \begin{tikzpicture}[level distance=1cm,level 1/.style={sibling distance=4cm},
              level 2/.style={sibling distance=2cm},
              level 3/.style={sibling distance=1.5cm}]
            \tikzstyle{every node}=[draw, circle, minimum size=7mm, inner sep=1mm]

            \node {45}
            child {node {40}
                child {node {17} {
                        child {node {14}}
                        child {node {19}}
                      }}
                child {node {43}}
              }
            child {node {48}
                child {node {46}}
                child {node {49}}
              };
          \end{tikzpicture}
        }
        \caption{}
        \label{arbreQ2}
      \end{figure}
    \end{subQuestion}

    \begin{subQuestion}
      Dessiner l'arbre représenté par la liste suivante \mintinlinepy{[0, 35, 30, 12, 20, 23, 10, 5, 2]}.
    \end{subQuestion}
  \end{Question}


  On fournit la fonction \mintinlinepy{f} prenant en argument un arbre donné sous la forme d'une liste \python\ ainsi que l'indice du nœud
  racine d'un de ses sous-arbres et renvoyant un nombre entier.
  \begin{minted}{python}
def f(arbre: list, i: int) -> int:
    if i >= len(arbre) or arbre[i] is None:
        return 0
    else:
        g = f(arbre, 2*i)
        d = f(arbre, 2*i + 1)
        return 1 + max(g, d)
    \end{minted}

  \begin{Question}
    \begin{subQuestion}
      On considère la variable \mintinlinepy{arbre} égale à \mintinlinepy{[0, 30, 20, 40, 18, 25, None, 47]}. Que renvoie l'appel \mintinlinepy{f(arbre, 1)} ? Justifier.
    \end{subQuestion}
    \begin{subQuestion}
      Que permet de calculer la fonction \mintinlinepy{f}?
    \end{subQuestion}
  \end{Question}

  \begin{Question}
    Compléter sur cet énoncé le code de la fonction suivante indiquant si un arbre est équilibré ou non. On rappelle que par définition un arbre vide est équilibré. \mintinlinepy{arbre} est la liste \python\ contenant un arbre et \mintinlinepy{i} l'indice d'un nœud, racine du sous-arbre étudié.
    \begin{minted}[baselinestretch=1.2, samepage]{python}
def estEquilibre(arbre: list, i : int) -> bool:
   if i >= len(arbre) or arbre[i] is None:
        return ......
    else:
        balance = f(arbre, .....) - f(arbre, .....)
        reponse = balance in [..., ..., ...]
        return reponse and estEquilibre(arbre,.....) and .........(.........,.....)
\end{minted}
  \end{Question}

  \noindent{\textbf{Partie B :}}
  \setcounter{question}{0}

  On s'intéresse dans cette sous-partie aux arbres binaires de recherche. On rappelle que dans un tel arbre, quel que soit le nœud considéré, son fils gauche, s'il existe, a une valeur
  qui lui est inférieure ou égale et son fils droit, s'il existe, une valeur qui lui est strictement supérieure. Les différents arbres dessinés plus haut sont tous des arbres
  binaires de recherche.

  \begin{Question}
    Donner les parcours \textit{préfixe}, \textit{infixe} et \textit{suffixe} de l'arbre de la figure \ref{arbreQ2}.
  \end{Question}

  Dans le cas où un arbre binaire de recherche n'est pas équilibré, la méthode ci-dessous permet de construire un arbre binaire de recherche équilibré
  contenant les mêmes valeurs :
  \begin{center}
    \fbox{
      \begin{minipage}{0.95\textwidth}
        \begin{enumerate}[label=\small{(\arabic*)}, leftmargin=*]
          \item Déterminer l'ordre de parcours \textit{infixe} de l'arbre. Le stocker dans une liste \mintinlinepy{ordre}
          \item Créer un arbre vide nommé \mintinlinepy{nouveau} (une liste contenant la valeur 0 selon l'implémentation choisie ici). La liste \mintinlinepy{nouveau} sera réutilisée dans la suite de 
          l'algorithme sans avoir besoin d'être passée en argument de fonctions
          \item La variable \mintinlinepy{i} prend la valeur 1 (indice de la racine de \mintinlinepy{nouveau})
          \item Appeler la fonction récursive \mintinlinepy{construireABR(i, ordre)} qui :
                \begin{enumerate}[label=\small{(\alph*)}]
                  \item Ajoute la valeur \mintinlinepy{None} à la liste \mintinlinepy{nouveau} jusqu'à ce qu'elle soit de longueur \mintinlinepy{i+1}
                  \item Donne la valeur du \og milieu \fg\ de \mintinlinepy{ordre} à \mintinlinepy{nouveau[i]}
                  \item Détermine la \mintinlinepy{moitié gauche de ordre}
                  \item Si celle-ci est non-vide, appelle \mintinlinepy{construireABR(2*i, moitié gauche de ordre)}
                  \item Détermine la \mintinlinepy{moitié droite de ordre}
                  \item Si celle-ci est non-vide, appelle \mintinlinepy{construireABR(2*i+1, moitié droite de ordre)}
                \end{enumerate}
        \end{enumerate}
      \end{minipage}
    }
  \end{center}

  La \mintinlinepy{moitié gauche de ordre} (resp. \mintinlinepy{moitié droite de ordre}) est la liste des valeurs de \mintinlinepy{ordre} d'indice
  strictement inférieur (resp. supérieur) à celui du milieu. On la crée facilement en faisant \mintinlinepy{ordre[:len(ordre)//2]}
  (resp. \mintinlinepy{ordre[(len(ordre)//2+1):]})

  Cette méthode, appliquée à l'ordre de parcours \mintinlinepy{[8, 10, 15, 17]} permet de construire l'arbre stocké dans la liste \mintinlinepy{[0, 15, 10, 17, 8]}.

  \begin{figure}[H]
    \centering
    \footnotesize{
      \begin{tikzpicture}[level distance=1.5cm,level 1/.style={sibling distance=3cm},
          level 2/.style={sibling distance=2cm},
          level 3/.style={sibling distance=1.5cm}]
        \tikzstyle{every node}=[draw, circle, minimum size=7mm, inner sep=1mm]

        \node (R) {15}
        child {node {10}
            child {node {8}}
            child {node[draw=none] {} edge from parent [draw=none]}
          }
        child {node {17}};

        \node[minimum size=7mm, inner sep=1mm, draw, rectangle, right=8cm of R, fill=lightgray!50] (150) {15};
        \node[minimum size=7mm, inner sep=1mm, draw, rectangle, left=0cm of 150] (100) {10};
        \node[minimum size=7mm, inner sep=1mm, draw, rectangle, left=0cm of 100] (80) {8};
        \node[minimum size=7mm, inner sep=1mm, draw, rectangle, right=0cm of 150] (170) {17};

        \node[minimum size=7mm, inner sep=1mm, draw, rectangle, fill=lightgray!50] (101) at ($(100)+(-1.2,-1.5)$) {10};
        \node[minimum size=7mm, inner sep=1mm, draw, rectangle, left=0cm of 101] (81) {8};
        \node[minimum size=7mm, inner sep=1mm, draw, rectangle, fill=lightgray!50] (171) at ($(170)+(1.2,-1.5)$) {17};

        \node[minimum size=7mm, inner sep=1mm, draw, rectangle, fill=lightgray!50] (82) at ($(81)+(-0.6,-1.5)$) {8};

        \draw[-latex, thick] (100.south west) -- (101.north west) node [midway, left, draw=none, rectangle, xshift=-1mm] {\footnotesize{\textit{moitié gauche}}};
        \draw[-latex, thick] (81.south) -- (82.north) node [midway, left, draw=none, rectangle] {\footnotesize{\textit{moitié gauche}}};
        \draw[-latex, thick] (170.south) -- (171.north) node [midway, right, draw=none, rectangle, xshift=1mm] {\footnotesize{\textit{moitié droite}}};
      \end{tikzpicture}
    }
    \caption*{}
  \end{figure}


  \begin{Question}
    Dessiner l'arbre construit par cette méthode lorsque l'ordre proposé est \mintinlinepy{[20, 21, 22, 23, 24, 25]}.
  \end{Question}

  On donne ci-dessous le pseudo-code d'une fonction renvoyant l'ordre de parcours infixe d'un arbre fourni sous forme d'une liste \python. Cette fonction utilise une pile.
  \begin{minted}{text}
Fonction infixe(arbre) :
    p est une Pile vide
    visites est une liste vide
    n est égal à 1 # l'indice du nœud étudié dans arbre
    repetition est égal à Vrai
    Tant que repetition est Vrai :
        Tant que n < longueur(arbre) et arbre[n] n'est pas None :
            Ajouter n à la pile
            n est égal à l'indice du fils gauche du nœud d'indice n
        Si la longueur de la pile est nulle :
            repetition est égal à Faux
        Sinon :
            Dépiler une valeur de la pile dans n
            Ajouter arbre[n] à la fin de visites
            n est égal à l'indice du fils droit du nœud d'indice n
    Renvoyer visites
\end{minted}
  \begin{Question}
    Transcrire cette fonction en \python. On pourra utiliser une liste \python\ afin d'implémenter la structure de pile.
  \end{Question}

  \begin{Question}
    Écrire en \python\ le code de la fonction \mintinlinepy{construireABR} décrite plus haut (point \small{(4)} de l'encadré).
  \end{Question}

\end{Exercice}

\end{document}